\documentclass[11pt]{article}

\setlength{\parindent}{0in}

\usepackage{graphicx}
\usepackage{amsmath}
\usepackage{mathrsfs}
\usepackage{textcomp}
\usepackage{setspace}
\usepackage{array,multirow}
\usepackage{tikz}

\begin{document}

\textbf{Review Notes for Binomial Option Pricing} \\
\textbf{Finance 5350: Computational Financial Modeling} \\
\textbf{Fall Semester 2017} \\
\vspace{5mm}

\bigskip
What follows is a derivation of the single--period Binomial option pricing formula.
This derivation is slightly different than the one found in your textbook.  I use different variable names than the text in order
to be more consistent with the Black--Scholes model.


\vspace{5mm}
To fix ideas, recall that our simple assumption of binomial prices leads to two
binomial trees: one for the stock price, and one for the option price:

\vspace{10mm}
\begin{center}
\tikzstyle{bag} = [text width=2em, text centered]
\tikzstyle{end} = []
\begin{tikzpicture}[sloped]
  \node (a) at ( 0,0) [bag] {$S_{0}$};
  \node (b) at ( 4,-1.5) [bag] {$dS_{0}$};
  \node (c) at ( 4,1.5) [bag] {$uS_{0}$};
  \draw [->] (a) to node [below] {} (b);
  \draw [->] (a) to node [above] {} (c);
\end{tikzpicture}
\end{center}

\vspace{10mm}
\begin{center}
\tikzstyle{bag} = [text width=2em, text centered]
\tikzstyle{end} = []
\begin{tikzpicture}[sloped]
  \node (a) at ( 0,0) [bag] {$C_{0}$};
  \node (b) at ( 4,-1.5) [bag] {$C_{d}$};
  \node (c) at ( 4,1.5) [bag] {$C_{u}$};
  \draw [->] (a) to node [below] {} (b);
  \draw [->] (a) to node [above] {} (c);
\end{tikzpicture}
\end{center}

\medskip
The basic idea of the Binomial Option Pricing Model is to set up a replicating portfolio to synthetically
replicate the European call option payoff. This lead to a simple equation:

\medskip
\begin{equation*}
C_{0} = \Delta S + B
\end{equation*}

\medskip
where $\Delta$ and $B$ are chosen with care so as to perfectly replicate the call option\footnote{The same logic
applies for put options, so we can talk only about call options without loss of generality.} This begs the
question: just how are $\Delta$ and $B$ chosen? We can solve for these parameters by noting that the following
must hold:

\medskip
\begin{eqnarray*}
 C_{u} &=& \Delta \times u S + B e^{rh} \\
 C_{d} &=& \Delta \times d S + B e^{rh} \\
\end{eqnarray*}

\medskip
We can now see how to solve for these parameters. First we will solve for $B e^{rh}$ in the second equation
as follows:

\medskip
\begin{equation*}
B e^{rh} = C_{d} - \Delta \times d S
\end{equation*}

\medskip
and plug it into the first for $B e^{rh}$ as follows:

\medskip
\begin{equation*}
C_{u} = \Delta \times u S + C_{d} - \Delta d S
\end{equation*}

\medskip
We notice that $B$ has now disappeared from the first equation and we can solve for $\Delta$ as follows:

\begin{equation*}
\Delta S (u - d) = C_{u} - C_{d}
\end{equation*}

\medskip
which leads to:

\medskip
\begin{equation*}
\Delta = \frac{C_{u} - C_{d}}{S(u-d)}
\end{equation*}

\medskip
So now we have solved for the correct value of $\Delta$ that gives us the number of shares we need to hold
in our portfolio to synthetically replicate the call option.  We can now plug this $\Delta$ into
$B e^{rh} = C_{d} - \Delta \times dS$ to get an equation, for which the only unknown is $B$ and solve for it.
We do this as follows:

\medskip
\begin{equation*}
B e^{rh} = C_{d} - \left(\frac{C_{u} - C_{d}}{S(u-d)}\right) \times d S
\end{equation*}

\medskip
which we can rearrange as:

\medskip
\begin{eqnarray*}
B e^{rh} &=& C_{d} \times \frac{(u-d)}{(u-d)} - \left(\frac{d C_{u} - d C_{d}}{u-d}\right) \\
         &=& \frac{u C_{d} - d C_{d} - d C_{u} + d C_{d}}{u-d} \\
         &=& \frac{u C_{d} - d C_{u}}{u-d}
\end{eqnarray*}

\medskip
Finally, we can multiply both sides of the equation by $e^{-rh}$ to get the following:

\medskip
\begin{equation*}
B = e^{-rh} \left( \frac{u C_{d} - d C_{u}}{u-d} \right)
\end{equation*}

\medskip
We now know what the values of $\Delta$ and $B$ need to be to perfectly replicate the call option. Since
we can observe these quantities, we can figure out by applying the \textbf{law of one price} (or in
other words by assuming no arbitrage opportunities exist) the equilibrium price of the call option
now, or $C_{0}$.

\medskip
We simply plug in for $\Delta$ and $B$ in the following:

\begin{eqnarray*}
C_{0} &=& \Delta \times S + B \\
      &=& \left( \frac{C_{u} - C_{d}}{S(u-d)} \right) \times S + e^{-rh} \left( \frac{u C_{d} - d C_{u}}{u-d} \right) \\
\end{eqnarray*}

\medskip
Essentially we could stop here.  We are done.  We have derived the single--period Binomial Option Pricing Model. But
we will keep working to rearrange this equation to express it in such a manner to get even more deep intuition from it.
We can rewrite the model as follows:

\medskip
\begin{eqnarray*}
C_{0} &=& \left( \frac{C_{u} - C_{d}}{S(u-d)} \right) \times S + e^{-rh} \left( \frac{u C_{d} - d C_{u}}{u-d} \right) \\
      &=& \left( \frac{C_{u} - C_{d}}{(u-d)} \right) + e^{-rh} \left( \frac{u C_{d} - d C_{u}}{u-d} \right) \\
      &=& e^{-rh} \left( \frac{e^{rh} C_{u} - e^{rh} C_{d} + u C_{d} - d C_{u}}{u-d} \right) \\
      &=& e^{-rh} \left(C_{u} \frac{e^{rh} - d}{u-d} + C_{d} \frac{u - e^{rh}}{u-d} \right)
\end{eqnarray*}

\medskip
Finally, we can let $p_{u}^{\ast} = \frac{e^{rh} - d}{u-d}$ and $p_{d}^{\ast} = \frac{u - e^{rh}}{u-d}$. Now we can write the
model simply as:

\medskip
\begin{equation*}
C_{0} = e^{-rh} \left[ C_{u} p_{u}^{\ast} + C_{d} p_{d}^{\ast} \right]
\end{equation*}


\end{document}
